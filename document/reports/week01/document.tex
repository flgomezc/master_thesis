\documentclass[preprint]{aastex62}
% \documentclass[manuscript]{aastex62}
%%  twocolumn, manuscript, preprint, preprint, modern and RNAAS
\usepackage[utf8]{inputenc}
\usepackage{lineno}
\newcommand{\vdag}{(v)^\dagger}
\newcommand\aastex{AAS\TeX}
\newcommand\latex{La\TeX}
%% Tells LaTeX to search for image files in the 
%% current directory as well as in the figures/ folder.
\graphicspath{{./}{figures/}}
%% Reintroduced the \received and \accepted commands from AASTeX v5.2
%%\received{January 1, 2018}
%%\revised{January 7, 2018}
%%\accepted{\today}
\shorttitle{Weekly Activity Report}
\shortauthors{Felipe L. G\'omez-Cort\'es}

%\renewcommand{\abstractname}{RESUMEN}
%\renewcommand{\figurename}{Figura}

\linenumbers
\begin{document}

\title{Weekly Activity Report\\Weeks 1-2\\A first algorythm comparing two $\beta$-Skeletons}

\correspondingauthor{Felipe Leonardo Gómez-Cortés}
\email{fl.gomez10@uniandes.edu.co}

\author{Felipe Leonardo Gómez-Cortés}
\affiliation{Physics Department, Universidad de Los Andes}
\collaboration{Master Student}

\nocollaboration

\author{Jaime E. Forero-Romero}
\affiliation{Physics Department, Universidad de Los Andes}
\collaboration{Advisor}


\begin{abstract}
  The beta parameter graph and its definition.
  
  The testing catalogs.

  How the graph changes while variying $\beta$.

  How this first attemp fails over very excentric ellipsoidal voids.
  
  \keywords{ Beta Skeleton Graph, Beta Parameter}

\end{abstract}

\section{The Algorythm}

The algorytm uses the NGL library % ######################(CITAR NGRAPH.ORG)
to create two $\beta$-Skeleton graph of a given set of points,
and comparing the diferences between them to identify the
voids in the structure of points.

The graphs of the 0.9-Skeleton and the 1-Skeleton are
calculated, then compared.
Each lenght is measured and the histogram of connection
lenghts is created for both $\beta$ values.
(in a similar fashion to the two-point correlation function)
By removing the short
connections -common in both skeletons-, and preserving the
long connections -those that cross througth the voids-,
the points envolving the voids are identified.


\subsection{Mock Catalog}
In order to test the algorythm, three catalogs of points in a 3D space
where created with empty regions emulating voids in the LSS.

The emulated space is a cubic region of 60 Mpc/h.
There where placed $\sim 5 \times10^4$ points to have a similar
volume density of points to the halo volumetric density in
the AbacusCosmos simulations. ($8.7\times10^6$ halos in a cubic
box of 720 Mpc/h length,
$2.335\times10^{-2}\mathrm{halo/(Mpc/h)^3}$).
Points where placed using an uniform density probability distribution.

Then, using the equation for the sphere and the ellipsoid, the subset
of points within the surface is removed from the catalog.

The first set of points has an spherical empty region of radius
20 Mpc/h centered in the middle of the volume.

The second catalog has four non-overlapping spherical empty
regions of radius between 8 and 20 Mpc/h.

The third catalog has four overlapping ellipsoidal voids.

\subsection{The $\beta$-Skeleton Graph}

The $\beta$-Skeleton graph is defined by the relative distance
between points (vertices) in a N-dimentional space, and a
geometrical criterion using a real
parameter $\beta\geq 0$. This graph is not directed, there is no
a preferential direction in the edges connecting vertices.

The geometrical criterion is the following:
two points ($p$,$q$) are connected in the graph if there is an empty region
between them, without any other point. The shape of the empty region
is function of the $\beta$/parameter and the distance $d(p,q)$ between the
points. In the case $\beta=0$ the empty region is a N-ball of
diameter $d$ centered in the midpoint between 

The NGraph authors define the $\beta$-skeleton as folows:

``The so-called lune-based $\beta$-skeleton is a one-parameter
generalization of the RNG [Relative Neighborhood Graph] and
GG [Gabriel Graph], defined as follows:

\begin{itemize}
\item For $0<\beta<1$, the empty region is the intersection of all
  d-balls with diameter $d(p,q) / \beta$ that have p and q on the
  boundary.
\item For $\beta \geq 1$, the empty region is the intersection of
  two d-balls with diameter $\beta d(p,q)$ centered at
  $(1-\frac{\beta}{2})p + \frac{\beta}{2}q$ and
  $\frac{\beta}{2}p + (1-\frac{\beta}{2})q$.''
\end{itemize}
  
In the limit when $\beta$ tends to zero, every point is conected
to each point on the set, it corresponds to the graph used in
the classic two-point correlation function. Each point has N
connections. (With N as the number of points in the set).

When $\beta$ is increasing, the number of connections per point
is reduced. The first connections to vanish are the longer ones,
while the short connections persists.

\section{Structure deppendence of $\beta$ parameter}

By the nature of its definition, the $\beta$-skeleton structure
changes with the continuous $\beta$ parameter, from a highly connected
structure (when $\beta \rightarrow 0$) to an empty graph when $\beta$ 
grows to large positive values. 

The Skeleton, the number of connections per point, and the length of
connections, they deppend also from the distribution of points in the
space. Is expected to have diferent values for the cosmic web, but is
necessary to have a testing catalog with well well known structures
to test the algorythm.

The testing set was generated using a uniform density probability to place
the points inside the box. There is a transition in the Skeleton of this
catalog when $\beta$ is close to 1, it is shown in the figure
\ref{beta_graphs}. Small $\beta$ values allows the large
connections in the skeleton. (a, b and c). When $\beta \geq 1$ the large
connections through the spherically uniform void vanish.

\begin{figure}
  \gridline{
    \fig{single_spherical_void/beta_0_70.png}{0.32 \textwidth}{(a)}
    \fig{single_spherical_void/beta_0_90.png}{0.32 \textwidth}{(b)}
    \fig{single_spherical_void/beta_0_99.png}{0.32 \textwidth}{(c)}}
  \gridline{
    \fig{single_spherical_void/beta_1_00.png}{0.32 \textwidth}{(d)}}
  \gridline{
    \fig{single_spherical_void/beta_3_00.png}{0.32 \textwidth}{(e)}
    \fig{single_spherical_void/beta_6_00.png}{0.32 \textwidth}{(f)}
    \fig{single_spherical_void/beta_9_00.png}{0.32 \textwidth}{(g)}}
  \caption{\label{beta_graphs}
    Skeleton Structure dependence of $\beta$ around
    a single void. Seven different values for $\beta$ (0.7, 0.9, 0.99, 1, 3, 6,
    and 9). A transition can be observed when $\beta \rightarrow 1^-$. }
\end{figure}

\section{Detecting the First Void}

The 0.9 and 0.99-Skeleton where calculated and compared
with the 1-Skeleton. The histograms of number of connections per
point (figure \ref{Fig_Connections_per_point}) show that the
1-Skeleton he nodes tend to have 7 connections, while for the 0.9
and 0.99-Skeletons he mean goes around 16 and 10 connections per
node respectively. This graph may be useful in advanced stages of
this project, it doesn't give more information than the Skeletons
where sucesfully constructed by the NGRAPH library.

\begin{figure}
  \gridline{
    \fig{single_spherical_void/connections_per_point_090_and_100skeleton.pdf}
        {0.45 \textwidth}{(a)}
    \fig{single_spherical_void/connections_per_point_099_and_100skeleton.pdf}
        {0.45 \textwidth}{(b)}}
  \caption{Connections per point. \label{Fig_Connections_per_point}}
\end{figure}

More information can be inferred by analyzing the histograms of
connection lengths (figure \ref{Fig_Length_connection}):
the 1-Skeleton has short connections, none of
them larger than the radius of the spherical void (R=20Mpc/h). But
0.9 and 0.99 Skeletons show a peak close to the diameter of the
void in the testing catalog.

\begin{figure}
  \gridline{
    \fig{single_spherical_void/connection_length_090_and_100skeleton.pdf}
        {0.45 \textwidth}{(a)}
    \fig{single_spherical_void/connection_length_099_and_100skeleton.pdf}   
        {0.45 \textwidth}{(b)}}
  \caption{Lenght connection for different $\beta$ values.
    \label{Fig_Length_connection}}
\end{figure}

\subsection{Selecting Surface Ponits}

Once the typical length for the 1-Skeleton is detected, a low limit
distance can be established to separate long and short distances.
\textbf{This $d_{cut}$ was hand-selected} as $d_{cut}=20 \mathrm{Mpc/h}$.
The histograms a and b in figure \ref{Fig_Length_connection} show a
second point close to the diameter of the void. By keeping the nodes with
connections around that value (40 Mpc/h), and discarding the short
connections, the surface enclosing the void appears
(figure \ref{3d_scatter_single_void}). This method is good enough
for a single spherical void in a random set of points.

\begin{figure}
  \gridline{
    \fig{single_spherical_void/3d_scatter_detected_void_090.pdf}
        {0.45 \textwidth}{(a)}
    \fig{single_spherical_void/3d_scatter_detected_void_099.pdf}   
        {0.45 \textwidth}{(b)}}
  \caption{Surface points enclosing the void choosen by comparing
    connection lenghts in the 0.90 and 0.99-Skeleton versus the
    1-Skeleton.
    \label{3d_scatter_single_void}}
\end{figure}

\subsection{Detecting Multiple Spherical Voids}

The testing catalog used has four non-overlapping spherical voids
with different radius (10, 10, 12 and 20 Mpc/h) in the same cubic box
of 60 Mpc/h length.

The connection lenght histograms show the short connections for
regular populated space (all the contribution for the 1-Skeleton)
and the large connections for $\beta<1$, with peaks around the
diameter of the voids, as expected.
(Figure \ref{Fig_Length_connection_4V})

The difference between the voids and the populated structure is visible
at $\beta=0.99$; three different clusters appears in the rigth panel.
But in the left panel, the two bells are overlapping. There is no a
clear difference between the low scale connections and connections
through voids. (May it be visible with another bin selection?)

Again, the limit between short and large connections is handmade
selected as $d_{cut} = 15 \mathrm{Mpc/h}$. 

\begin{figure}
  \gridline{
    \fig{four_spherical_voids/connection_length_4voids_090_and_100skeleton}
        {0.45 \textwidth}{(a)}
    \fig{four_spherical_voids/connection_length_4voids_099_and_100skeleton}     
        {0.45 \textwidth}{(b)}}
  \caption{Lenght connection for different $\beta$ values with four
    voids inside the structure. Diameter of the voids: 20, 20, 24 and 40
    Mpc/h.
    \label{Fig_Length_connection_4V}}
\end{figure}

With this handmade selection of long connections, the algorythm sucesfully
identifies the four voids (figure \ref{3d_scatter_four_voids}).

\begin{figure}
  \gridline{
    \fig{four_spherical_voids/3d_scatter_detected_4voids_090.pdf}
        {0.45 \textwidth}{(a)}
    \fig{four_spherical_voids/3d_scatter_detected_4voids_099.pdf}
        {0.45 \textwidth}{(b)}}
  \caption{Surface points enclosing the four void choosen by comparing
    connection lenghts in the 0.90 and 0.99-Skeleton versus the
    1-Skeleton.
    \label{3d_scatter_four_voids}}
\end{figure}



\section{The Ellipsoidal Catastrophe}

Triying to give a leap, instead going step by step, the third catalog of points
includes three connected ellipsoidal voids, variying the coefficients in the
standard ellipsoid equation between 0.5 and 2.0. The algorytm fails to identify
the ellipsoidal voids, but detects the spherical control void.

\begin{figure}
  \gridline{
    \fig{ellipsoidal_catastrophe/ConnectionLenghtHistogram_for_090-100-skeletons.pdf}
        {0.45 \textwidth}{(a)}
    \fig{ellipsoidal_catastrophe/ConnectionLenghtHistogram_for_099-100-skeletons.pdf}
        {0.45 \textwidth}{(a)}}
  \gridline{
    \fig{ellipsoidal_catastrophe/ConnectionLenghtHistogram_for_07-13-skeletons.pdf}
        {0.45 \textwidth}{(c)}
    \fig{ellipsoidal_catastrophe/ConnectionLenghtHistogram_for_08-13-skeletons.pdf}
        {0.45 \textwidth}{(d)}}
  \caption{Lenght connection for different $\beta$ values with four
    connected ellipsoidal voids. There is no a clear difference in the histograms
    about connection lenght in populated space and empty regions.
    Upper Panels: using the previous $\beta$
    diference comparing aginst 1-Skeleton, it worked for spheroidal voids, but
    not for connected ellipsoidal voids.
    Lower panels: increasing the $\beta$ difference, comparing against 1.3-Skeleton.
    \label{Fig_Length_connection_Ellipsoids}}
\end{figure}


\begin{figure}
  \gridline{
    \fig{ellipsoidal_catastrophe/IrregularVoidSurface_for_090-100-skeletons.pdf}
        {0.45 \textwidth}{(a)}
    \fig{ellipsoidal_catastrophe/IrregularVoidSurface_for_08-13-skeletons.pdf}
        {0.45 \textwidth}{(b)}}
  \caption{Surface points enclosing the four connected ellipsoidal voids
    choosen by comparing connection lenghts in the 0.90-Skeleton versus the
    1-Skeleton (left panel). There is a notable lack of points in the upper
    and lower regions of the ellipsoids, as the front and the bottom. The algorytm
    fails to identify the spheroidal voids in the structure.
    In the right pannel there is an intent to fix the issue using a larger
    difference in the $\beta$ values, using the 0.8 and 1.3-Skeletons. The algorythm
    stills without identify the voids.
    \label{3d_scatter_four_voids}}
\end{figure}



\section{Conclusions}

An important factor as $d_{cut}$ cannot be arbitrarily selected. 

A diferent method is going to be implemented.

  \nocite{*}

  \begin{thebibliography}{20}

    %\bibitem[Bos et al.(2012)]{Bos2012} Bos, P. et al. \ 2012, \mnras, 426, 440 % Testing cosmologies using void ellipticity

  \end{thebibliography}                                                           
                       

%\listofchanges

\end{document}

 \documentclass[manuscript]{aastex62}
% \documentclass[manuscript]{aastex62}
%%  twocolumn, manuscript, preprint, preprint, modern and RNAAS
\usepackage[utf8]{inputenc}
%\usepackage{siunitx}
%\usepackage[spanish]{babel}
%
\newcommand{\vdag}{(v)^\dagger}
\newcommand\aastex{AAS\TeX}
\newcommand\latex{La\TeX}
%% Tells LaTeX to search for image files in the 
%% current directory as well as in the figures/ folder.
\graphicspath{{./}{figures/}}
%% Reintroduced the \received and \accepted commands from AASTeX v5.2
%%\received{January 1, 2018}
%%\revised{January 7, 2018}
%%\accepted{\today}
\shorttitle{A LSS Void Identifier based on $\beta$-Skeleton}
\shortauthors{F. L. Gómez-Cortés}

\renewcommand{\abstractname}{RESUMEN}
%\renewcommand{\thebibliography}{RESUMEN}

\renewcommand{\figurename}{Figura}


\begin{document}

\title{Proyecto de investigación:\\``A Large Scale Structure Void Identifier for Galaxy Surveys
  Based on the $\beta$-Skeleton Graph Method'' \\en el marco de la convocatoria\\
  2019-20 - PARA LA FINANCIACIÓN DE PROYECTOS DE INVESTIGACIÓN Y PRESENTACIÓN DE
  RESULTADOS EN EVENTOS ACADÉMICOS CATEGORÍA: ESTUDIANTES DE MAESTRÍA Y
  DOCTORADO ANTES DE EXAMEN DE CANDIDATURA}

  
\correspondingauthor{Felipe Leonardo Gómez-Cortés}
\email{fl.gomez10@uniandes.edu.co}

\author{Felipe Leonardo Gómez-Cortés}
\affiliation{Departamento de Física, Universidad de los Andes}
\collaboration{Estudiante de Maestría en Ciencias-Física}
\collaboration{Código 201324084}



\section{Investigadores}

\begin{centering}
\begin{table}[h]
\begin{tabular}{lp{6.5cm}p{5cm}}
\hline
\hline
Felipe Leonardo Gómez Cortés & Estudiante Maestría en Ciencias-Física, Un (1) semestre faltante para el grado & Universidad de los Andes      \\
Jaime Ernesto Forero Romero  & Director de Tesis, Profesor Asociado, Departamento de Física & Universidad de los Andes      \\ 
Xiao-Dong Li                 & Colaborador                                                  & Sun Yat Sen University, China \\ \hline \hline
\end{tabular}
\end{table}
\end{centering}

\section{Propuesta}

\subsection{Título}
A Large Scale Structure Void Identifier for Galaxy Surveys Based on the $\beta$-Skeleton Graph Method

\subsection{Resumen}
Large underdense regions in the Large Scale Structure (LSS) of the Universe, also known as voids,
are a prominent features of the cosmic web that can also be used as a cosmological probe.
This project presents a new void-finding algorithm that can be applied
to both observational and simulated data.
The algorithm is based on the $\beta$-Skeleton, an algorithm widely used on 
machine learning, optimization and image recognition;  recently it has been introduced as a 
LSS analysis tool.
The analysis we have performed, on observational and simulated data, considers voids as ellipsoids.
We study their statistical properties such as volumes and shape finding a good agreement with other void finders.
We finalize by exploring possible applications of this void finder to constrain cosmological
parameters based on data from the Dark Energy Spectroscopic Instrument.

\subsection{Introducción}

Estamos viviendo en la era dorada de la cosmolog\'ia observacional.
Existe un modelo est\'andar comol\'ogico ($\Lambda$-CDM) consolidado 
que explica las observaciones de la Estructura de Gran Escala (EGE) de galaxias mediante
la introducci\'on de materia oscura y energ\'ia oscura como las componentes
dominantes del Universo.

Las observaciones del fondo de radiaci\'on c\'osmica de microondas
\citep{WMAP2013} y de la distribuci\'on de galaxias a gran escala
\citep{SDSS-DR14-2017} apuntan a que esta evoluci\'on puede ser descrita
por un pu\~nado de par\'ametros cosmol\'ogicos, donde los m\'as
importantes son la densidad de materia y la densidad de energ\'ia
oscura. 

Esto no s\'olo se logra con mediciones m\'as precisas sino con
m\'etodos independientes para acotar los par\'ametros
cosmol\'ogicos.
Aunque un m\'etodo independiente pueda tener una incertidumbre
grande, considerar las cotas impuestas por varios m\'etodos
simult\'aneamente reduce la incertidumbre sobre los par\'ametros
cosmol\'ogicos.


Una de las caracter\'isticas m\'as prominentes en la EGE son los vac\'ios: vol\'umenes
irregulares de escalas del orden de decenas de Mpc, donde la densidad de materia est\'a
por debajo de la densidad media en el Universo. El an\'alisis estad\'istico de propiedades
de los vac\'ios, como su volumen, forma y orientaci\'on tambi\'en nos puede dar informaci\'on
cosmol\'ogica. Por esta raz\'on existe un gran inter\'es en algoritmos que encuentren y
caractericen vac\'ios cosmol\'ogicos tanto en simulaciones como en observaciones.

El m\'etodo $\beta$-Skeleton ha sido ampliamente utilizado
en procesamiento de im\'agenes y aplicaciones de \textit{machine learning},
recientemente ha sido introducido en el an\'alisis de EGE. Esta es una herramienta r\'apida
para identificar estructuras filamentarias en la EGE, y promete ser una herramienta robusta
para realizar an\'alisis cosmol\'ogicos.

En este proyecto se desarrolla un buscador de vacíos cosmológicos basado en el
$\beta$-Skeleton para calcular restricciones sobre los parámetros cosmológicos en el
modelo $\Lambda$-CDM


\begin{figure}
  \plotone{pie_millennium_walls}
  \caption{Estructura de Gran Escala: Distribuci\'on espacial de galaxias observada en mapeos como el SDSS y
    el 2dFRGS (en azul) comparadas con resultados de la simulación Millenium (rojo).
    Instituto de Astrofísica Max Planck. Cada punto representa una galaxia.
    La Vía Láctea se encuentra en el centro de los mapas, que cubren diferentes regiones
    del cielo en escalas distintas.
    En el segmento de la izquierda muestra la estructura a gran escala compuesta por
    filamentos donde se agrupan las galaxias y grandes vacíos donde es baja la densidad
    de galaxias, del catálogo 2dFGRS. Hacia la periferia distante solo se pueden ver las
    galaxias más brillantes, por eso la baja densidad hacia el borde.
    A la derecha vemos en un segmento del la simulación Millenium,
    En los
    segmentos superiores vemos estructuras de varias decenas de Mpc encontradas en dos
    catálogos (SDSS y 2dFGRS). En los dos segmentos inferiores, vemos estructuras con
    tamaños similares generadas por la misma simulación Millenium.
    \label{fig:pie_millenium_walls}}
\end{figure}


\subsection{Objetivos}
\begin{itemize}
\item Desarrollar un nuevo identificador de vac\'ios cosmológicos basado en el m\'etodo
  $\beta$-Skeleton que opere tanto sobre catálogos de galaxias observadas como en
  catálogos provenientes de simulaciones cosmológicas.
\item Buscar vacíos sobre catálogos de galaxias del SDSS y catálogos de la simulación
  Abacus
\item Comparar estadísticamente morfología y abundancia de los vacíos encontrados con
  nuestro buscador contra resultados de la literatura

\item Acotar los parámetros cosmológicos asociado a la densidad de materia y de
  energía oscura a partir del catálogo de vacíos de la red cósmica
\item Realizar predicciones para la población de vacíos cosmológicos en el proyecto
  DESI basado en el análisis de simulaciones de gran tamaño con nuestro buscador de vacíos.
\end{itemize}

\subsection{Resultados Esperados}

Ya se ha desarrollado el código de un buscador de vacíos de la red cósmica basado en el
método $\beta$-Skeleton. Se tienen resultados preliminares del análisis morfológico
sobre una fracción del catálogo de galaxias del SDSS.

Esperamos realizar pruebas cosmológicas tipo Alcock-Paczynski \citet{AlcockPaczynski1979}
para acotar el parámetro cosmológico de Energía Oscura
y pruebas morfológicas de elipticidad y relación entre semi-ejes,
\citep{Bos2012,Lavaux-Wandelt2009,Park-Lee2007}, sobre los halos encontrados en
catálogos de galaxias de simulaciones y observaciones, y finalmente
estimar la población de vacíos que encontrará el proyecto DESI.

\subsection{Productos Esperados}
Este trabajo será presentado como ponencia en el evento académico \textit{LARIM 2019: XVI
  Latin American Regional IAU Meeting} organizado por la Unión Astronómica
Internacional, Noviembre 3-8 de 2019, Antonfagasta, Chile.

\subsection{Literatura Citada}


\subsection{Cronograma de Actividades}
\begin{table}[h]
  \centering
  \begin{tabular}{|c|c|c|c|c|c|c|c|} \hline\hline
    Actividad/Mes & Feb-May & Junio & Julio & Agosto & Septiembre & Octubre & Noviembre \\ \hline
    1 & Hecho &   &   &   &   &   &    \\
    2 &   & En Curso &   &   &   &   &    \\
    3 &   &   & X &   &   &   &    \\
    4 &   &   & X &   &   &   &    \\
    5 &   &   & X & X &   &   &    \\
    6 &   &   &   & X & X &   &    \\
    7 &   &   &   &   & X & X &    \\
    8 &   &   &   &   &   &   & X \\
    9 &   &   &   &   &   &   & X \\
   10 &   &   & X & X & X & X & X \\    \hline
\end{tabular}
\end{table}

\begin{enumerate}
\item Desarrollo del Código del Identificador de Vacíos de la Cosmológica
\item Calibración de los parámetros del código.
\item Obtención de catálogos de vacíos de la red a partir de simulaciones.
\item Obtención de catálogos de vacíos de la red a partir de observaciones.
\item Comparación por análisis estadístico entre simulaciones y observaciones.
\item Restricción de la constante cosmológica a partir de catálogo de vacíos de la red en
    observaciones.
\item Estimación de resultados para el experimento DESI.
\item Presentación de la Ponencia en LARIM 2019
\item Sustentación de Tesis
\item Elaboración del manuscrito del Artículo
\end{enumerate}

\subsection{Presupuesto}

De acuerdo a las condiciones de la convocatoria, se destinará la financiación para
participar con una ponencia en el evento académico LARIM 2019, Chile, del 3 al 8 de
noviembre de 2019.
Se solicita un presupuesto de \$3'366.000 COP
Los gastos estimados
se muestran a continuación.

\begin{table}[h]
\centering
\begin{tabular}{lr}
Rubro                               & Valor  \\ \hline
Inscripción a LARIM 2019 (100 EUR)  &   \$366.000 COP   \\
Tiquetes BOG-ANF                    & \$2'000.000 COP \\
Alojamiento  (7 días)               &   \$300.000 COP \\
Alimentación y Transporte interno   &   \$700.000 COP \\ \hline
TOTAL                               & \$3'366.000 COP \\
\end{tabular}
\end{table}


\section{Consideraciones Éticas}
Este proyecto no incluye sujetos de investigación humanos.

\section{Uso de Animales y Colectas}
Este proyecto no incluye como sujetos de investigación seres vivos.

  \nocite{*}

  \begin{thebibliography}{}

  \bibitem[Alcock-Paczynski(1979)]{AlcockPaczynski1979} Alcock, C. \& Paczy\'nski, B.\ 1979, \nat, 281, 358    
    \bibitem[Colberg et al.(2008)]{Aspen-Amsterdam2008} Colbert, J. M. et al. \ 2008, \mnras, 387, 933. % Comparison of multiple void finders 2008

    \bibitem[Bos et al.(2012)]{Bos2012} Bos, P. et al. \ 2012, \mnras, 426, 440 % Testing cosmologies using void ellipticity
    \bibitem[Bustamante \& Forero-Romero (2015)]{Bustamante2015} Bustamante, S \& Forero-Romero, J. E. \ 2015, \mnras, 453, 497-506
    \bibitem[Correa \& Lindstrom(2012)]{ngl} Correa, Carlos \& Lindstrom, Peter.\ 2011,  IEEE TVCG\ 17,12 (Dec 2011), 1852-1861
    \bibitem[El-Ad \& Piran(1997)]{El-Ad1997} El-Ad, H. \& Piran, T. \ 1997, \apj, 491, 2, 421
    \bibitem[Fang et al.(2018)]{Fang2018} Fang, F.; Forero-Romero, J.; Rossi, G.; Li, X. \& Feng, L\ 2018, arXiv, 1809.00438 astro-ph % Beta-Skeleton analysis of the Cosmic Web
    \bibitem[Hamaus et al.(2015)]{Hamaus2015} Hamaus, N.; Sutter, P.M.; Lavaux, G. \& Wandelt, B. D. \ 2015, \jcap, 11, 036    
    \bibitem[Lavaux \& Wandelt(2009)]{Lavaux-Wandelt2009} Lavaux, G. \& Wandelt, B. D. \ 2009, \mnras, 403, 1392
    \bibitem[Leclercq et al.(2015)]{2015JCAP...03..047L} Leclercq, F.; Jasche, J. et al.\ 2015, \jcap, 03, 047
    \bibitem[Park \& Lee(2007)]{Park-Lee2007} Park, D. \& Lee, J. \ 2007, \prl 98, 1301 % Void Ellipticity as a probe of Cosmology
    \bibitem[Press \& Schechter(1974)]{Press&Schechter1974} Press, W. H. \& Schechter, P. \ 1974, \apj, 187, 425-438
    \bibitem[Riebe et al.(2013)]{Multidark2013} Riebe, K. et al \ 2013, Astronomical Notes, 334, 691 % Bolshoi Simulation      
    \bibitem[Schneider(2014)]{Schneider2014} Schneider, P. \ 2014, ``Extragalactic Astronomy and Cosmology'', Springer
    \bibitem[Smith et al.(2017)]{Smith2017} Smith, A. et al. \ 2017, \mnras, 470, 4646 % Lightcone Millenium MXXL
    \bibitem[Smith et al.(2018)]{Smith2018} Smith, A. et al. \ 2018, arXiv, 1809.07355 astro-ph % Fibre assignament incompleteness in DESI
    \bibitem[Springel et al.(2005)]{Springel2005} Springel, V. et al. \ 2005, \nat, 435, 639 % Millenium Simulation
    \bibitem[Sutter et al.(2015)]{Sutter2015} Sutter, P. M.; Lavaux G.; Hamaus, N; et al. \ 2015, A\&C, 9, 1-9 % Void Finder VIDE
    \bibitem[van de Weygaert(2014)]{Weygaert2014} van de Weygaert, Rien\ 2014, Proceedings of the IAU, 308, 493 % Review of Voids

    \bibitem[SDSS Collaboration(2017)]{SDSS-DR14-2017} SDSS Collaboration \ 2017, arxiv, 1707.09322 astro-ph % SDSS Data Release 14, 2017
    \bibitem[Hinshaw et al.(2013)]{WMAP2013} Hinshaw, G. et al. \ 2013, \apjs, 208, 20. % WMAP last release
  \end{thebibliography}                                                           
                       





  

\end{document}



%\listofchanges


% \citet{https://arxiv.org/pdf/astro-ph/0610520.pdf} Bayes for SDSS EGE reconstruction

% \bibitem[Aarseth(2003)]{Aarseth2003} Aarseth, S. J. \ 2003, ``Gravitational N-Body Simulations'', Cambridge University Press. % Book
% \bibitem[Longair(2004)]{Longair2004} Longair, M. S. \ 2004, ``A Brief History of Cosmology'', ``Measuring and Modeling the Universe'', Carnegie Observatories Astrophysics Series, Vol 2. % BOOK

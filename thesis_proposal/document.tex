\documentclass[manuscript]{aastex62}
% \documentclass[manuscript]{aastex62}
%%  twocolumn, manuscript, preprint, preprint, modern and RNAAS
\newcommand{\vdag}{(v)^\dagger}
\newcommand\aastex{AAS\TeX}
\newcommand\latex{La\TeX}
%% Tells LaTeX to search for image files in the 
%% current directory as well as in the figures/ folder.
\graphicspath{{./}{figures/}}
%% Reintroduced the \received and \accepted commands from AASTeX v5.2
%%\received{January 1, 2018}
%%\revised{January 7, 2018}
%%\accepted{\today}
\shorttitle{A LSS Void Identifier based on $\beta$-Skeleton}
\shortauthors{F. L. G\'omez-Cort\'es}


\begin{document}

\title{A Large Scale Structure Void Identifier for Galaxy Surveys
  Based on the $\beta$-Skeleton Graph Method}

\correspondingauthor{Felipe Leonardo G\'omez-Cort\'es}
\email{fl.gomez10@uniandes.edu.co}

\author{Felipe Leonardo G\'omez-Cort\'es}
\affiliation{Physics Department, Universidad de Los Andes}
\collaboration{(Master Student)}

\nocollaboration


\author{Jaime E. Forero-Romero}
\affiliation{Physics Department, Universidad de Los Andes}
\collaboration{(Advisor)}


\begin{abstract}

  We are living the golden age of cosmology. Since the three past decades
  we are able do precise measurements of cosmological parameters by
  observational methods. Besides, computational astrophysics has
  won its own place as the tool to probe theoretical models and compare
  them with observations.

  The standard cosmological model ($\Lambda$CDM) explains the observed
  Large Scale Structure (LSS) of galaxies by introducing dark matter and
  dark energy as the Universe components along with baryonic matter.
  The LSS has been reproduced in Gravitational N-bodies simulations
  such as Millenium and Bolshoi.

  One of the LSS elements are the voids; irregular huge volumes at $h^{-1}$Mpc
  scales, where the matter density is below the $20\%$ of the Universe
  average density and sparse non-massive galaxies. Voids are key
  elements to study Dark Energy and give important hints about other
  cosmological parameters. Statistics about voids population such as
  mass, shape and orientation encloses that information.

  The $\beta$-Skeleton method has been widely used on image processing,
  recognition and machine learnig applications, has been introduced
  recently in LSS analysis. It is a fast tool identifiying LSS filaments,
  and promises to be a robust tool to make statistical analysis
  as the two point correlation function and the Alcock-Pazcynski test,
  booth methods are currently used in cosmology.

  The objetive of this work is to develope a LSS void identifier based
  on the $\beta$-Skeleton method in order to improve the statistical
  analysis of voids in galaxy surveys and the constriction of
  cosmological parameters.


  
\end{abstract}

\keywords{ Large Scale Structure, cosmology, voids, computational astrophysics}


\section*{}
\nocite{*}
%% of citation expression problems, but it is necessary to clearly
%% delimit the year from the author name used in the citation.
%% See the natbib documentation for more details and options.




\begin{thebibliography}{}                                                       

\bibitem[van de Weygaert(2014)]{2016IAUS..308..493V} van de Weygaert, Rien\ 2014, Proceedings of the IAU, 308, 493   
\bibitem[Fang(2018)]{arXiv:1809.00438} Fang, F.; Forero-Romero, J.; Rossi, G.; Li, X. \& Feng, L\ 2018, arXiv, 1809.00438 astro-ph
\bibitem[Alcock-Paczynski(1979)]{1979Natur.281..358A} Alcock, C. \& Paczy\'nski, B.\ 1979, \nat, 281, 358
\bibitem[Leclercq(2015)]{2015JCAP...03..047L} Leclercq, F.; Jasche, J. et al.\ 2015, \jcap, 03, 047
                                                                                
\end{thebibliography}                                                           
                       


%% This command is needed to show the entire author+affilation list when
%% the collaboration and author truncation commands are used.  It has to
%% go at the end of the manuscript.
%\allauthors

%% Include this line if you are using the \added, \replaced, \deleted
%% commands to see a summary list of all changes at the end of the article.
%\listofchanges

\end{document}

% End of file `sample62.tex'.
